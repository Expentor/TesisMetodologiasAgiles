Las metodologías actuales si bien, son muy buenas y han mostrado que en su gran mayoría
funcionan, siempre existen desperfectos que la hacen que no funcionen al 100\% para todas
las personas, el objetivo de esta tesis es determinar qué conjunto de herramientas y formas
de trabajar (marco metodológico) es el ideal para que pueda existir una mejor comunicación
entre los clientes, programadores y project managers, así mismo, prometiendo entregar
software de calidad, de tal forma que cada parte del proyecto se sienta realizada y
satisfecha en cada punto del ciclo de vida del software, tratando de cumplir con todas las
expectativas que tengan cada parte de los equipos que involucran un proyecto de desarrollo
de software.