\section{\textbf{Capítulo 1:} Antecedentes y dar conocimiento sobre las metodologías existentes.}

	\subsection{Investigación de las metodologías actuales} 
		Para poder comenzar esta investigación es necesario tener claro cada parte de su contexto, por lo que es necesario realizar una investigación previa para conocer a detalle los antecedentes de las preguntas que se plantean.
		Se investigarán las metodologías tradicionales y ágiles que envuelven la ingeniería de
		software.
		
	\subsection{Análisis de requerimientos}
		Para poder finalizar la investigación es necesario saber cuál es la finalidad de esta, por lo que se hará uso de formularios de google y entrevistas a las partes interesadas para así conocer las necesidades y objetivos de esta investigación.
		
	\subsection{Elaboración de la tesina}
		Se elaborará un documento como anteproyecto de la investigación para dar a conocer todos los puntos abordados en la misma, así como dar lugar a las actividades que se realizarán bajo supervisión y asesoramiento de los docentes asignados por la facultad competente.
		En este documento se pretende demostrar que los estudiantes involucrados
		poseemos las capacidades necesarias para realizar está investigación.
		Este documento carece de un presupuesto ya que la investigación es meramente teórica.
		
\section{\textbf{Capítulo 2:} Proponer pasos generales para una nueva metodología ágil}

	\subsection{Análisis y comparación de las metodologías}
		Se tomarán en cuenta todas las metodologías existentes y variaciones de ellas para así poder tomar los puntos fuertes de todas estas.
	
	\subsection{Toma de puntos clave para el desarrollo de software}
		Aquí una vez tomados los puntos fuertes de cada metodología se buscarán los pasos y se documentarán con palabras clave cada uno de ellos de manera que se de a entender de manera clara y sencilla una descripción.
		
	\subsection{Elaborar diferentes propuestas para el desarrollo de una nueva metodología}
		Haciendo uso de las palabras clave del punto anterior, se crearán distintas propuestas uniendo dichas palabras clave de manera lógica y ordenada para así tener un panorama más claro de lo que pudiera ser un resultado final.
		
\section{\textbf{Capítulo 3:} Desarrollar una metodología innovadora para el desarrollo de software}

	\subsection{Elaboración de esquemas y diagramas de flujo para agilizar el proceso de desarrollode software}
	
		Una vez realizadas las distintas propuestas se tomarán en cuenta cada una de ellas para así generar material visual y conocer cuáles de ellas llegan a tener coherencia y a optimizar el tiempo de desarrollo del software. Gracias a estos materiales podemos reacomodar puntos clave y emerger nuevas propuestas.
	
	\subsection{Hacer una comparación de las diferentes propuestas para hacer un descarte}
		Se tomarán en cuenta todas las propuestas finales para así determinar cuáles son aquellas que se acerquen mas a los objetivos establecidos.
		
	\subsection{Elaborar un esquema con las palabras clave de la nueva metodología}	
		Se hará uso de material visual para generar una serie se pasos lógicos y secuenciales de los pasos de la nueva metodología.
		
	\subsection{Estructurar un marco metodológico}
		Elaborará un documento en que se de una descripción general al igual que se detallen los pasos a seguir de la nueva metodología. Cada paso tendrá una descripción que le ayude a los equipos de trabajo a acoplarse y entender cada uno de ellos.
		
\section{\textbf{Capítulo 4:} Pruebas}
	Para este capítulo se solicitarán equipos de software voluntarios para el uso de la nueva metodología, se validará que se haga uso de esta y se siga cada paso al pie de la letra para así comprobar que se cumplen los objetivos de la investigación.
	
