El desarrollo de software es un proceso complejo que se debe de realizar con precisión
planificando cada paso para lograr un resultado que cumpla con las expectativas
solicitadas. A lo largo del tiempo, se han creado diversas metodologías para guiar este
proceso, cada una con sus propias ventajas y desventajas.
Las metodologías tradicionales, como el modelo en cascada o el modelo en espiral, se
caracterizan por ser rígidas y secuenciales. Estas metodologías pueden ser útiles para
proyectos bien definidos y con requisitos estables, pero no son tan flexibles para adaptarse
a cambios o nuevas necesidades.
Por otro lado, las metodologías ágiles, como Scrum o Kanban, se basan en un enfoque
incremental e iterativo. Estas metodologías son más flexibles y adaptables a cambios, pero
pueden requerir una mayor disciplina y compromiso por parte del equipo de desarrollo.
Dado todo esto, se pretende crear una nueva metodología dirigida al desarrollo de software
que ayude a los equipos de trabajo a llevar a cabo las tareas de manera eficiente, escalable
y flexible combinando las mejores prácticas de las metodologías tradicionales y ágiles.