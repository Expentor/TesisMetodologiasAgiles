Se pretende crear una metodología ágil en base al continuo despliegue, para así evitar
problemas de entrega del proyecto con los clientes. Se mostrarán versiones preliminares en
etapas tempranas del proyecto, creemos que este feedback continuo nos ayudará a
entender sus necesidades en cada fase del proyecto. Para conseguir esto se necesita tener
una comunicación directa entre los desarrolladores y los clientes.
Realizamos una encuesta a personas que desempeñan deberes en una empresa de
desarrollo de software, la mayoría de las respuestas vienen de los programadores de las
empresas; con esto, se puede concluir que la gran mayoría de los involucrados en esta
encuesta buscan que la metodología sea aún más específica con el objetivo de estandarizar más los procesos, también lo que se puede rescatar de esta encuesta es que la gran
mayoría de las personas utilizan Scrum en su empresa seguido de la metodología Kanban,
sin lugar a dudas Scrum es un estándar en la industria, no obstante, otra idea que se puede
recuperar de esta encuesta es que, todos los encuestados no tendrían problemas en
aprender o aplicar una nueva metodología siempre y cuando prometa tener mejores
resultados y mejore la comunicación del equipo sin importar en qué área o qué tipo de
tecnologías estén utilizando.