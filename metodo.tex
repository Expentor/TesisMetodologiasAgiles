La finalidad de este semestre es iniciar el progreso de nuestra tesis. Desde la asignación del
tema en la primera semana, nos centramos en establecer la metodología de trabajo, investigar
antecedentes y elaborar un cronograma de tareas, evaluando también el tiempo y desempeño
proyectados para el proyecto. Se organizaron reuniones con los miembros del equipo de tesis, tanto
presenciales como virtuales, para facilitar la comunicación, alcanzar acuerdos y tomar decisiones
conjuntas.
En las etapas iniciales, se priorizó la recopilación de información relevante y pertinente para
el anteproyecto, incluyendo artículos y sitios web. Simultáneamente, se comenzó a elaborar el
reporte preliminar y a seleccionar las herramientas esenciales para el desarrollo del proyecto,
definiendo así los roles específicos de cada miembro.
Para la segunda semana, concluimos el desarrollo del anteproyecto, establecimos los
requerimientos específicos y delineamos el enfoque del sistema de préstamos. Emplearemos
tecnologías avanzadas y herramientas colaborativas para el desarrollo del sistema automatizado de
gestión de préstamos y devoluciones en la Facultad de Ingeniería Electromecánica. La
infraestructura tecnológica se basará en Node.js y Express.js para el backend, React Native para la
aplicación móvil, y MySQL para la base de datos, asegurando una arquitectura sólida y escalable.
Docker facilitará la contenerización para un despliegue eficiente.
Las herramientas de desarrollo incluirán VS Code, Git y GitHub para el control de
versiones y colaboración, Postman para el testeo de APIs, y phpMyAdmin para la gestión de la base
de datos. Canva y Figma se utilizarán para el diseño de la interfaz, y WhatsApp para la
comunicación del equipo. La implementación de estas tecnologías, en el marco de la metodología
Extreme Programming (XP), permitirá crear un sistema que satisfaga efectivamente las necesidades
del proyecto.